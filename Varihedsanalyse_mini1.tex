% Options for packages loaded elsewhere
\PassOptionsToPackage{unicode}{hyperref}
\PassOptionsToPackage{hyphens}{url}
%
\documentclass[
]{article}
\usepackage{lmodern}
\usepackage{amssymb,amsmath}
\usepackage{ifxetex,ifluatex}
\ifnum 0\ifxetex 1\fi\ifluatex 1\fi=0 % if pdftex
  \usepackage[T1]{fontenc}
  \usepackage[utf8]{inputenc}
  \usepackage{textcomp} % provide euro and other symbols
\else % if luatex or xetex
  \usepackage{unicode-math}
  \defaultfontfeatures{Scale=MatchLowercase}
  \defaultfontfeatures[\rmfamily]{Ligatures=TeX,Scale=1}
\fi
% Use upquote if available, for straight quotes in verbatim environments
\IfFileExists{upquote.sty}{\usepackage{upquote}}{}
\IfFileExists{microtype.sty}{% use microtype if available
  \usepackage[]{microtype}
  \UseMicrotypeSet[protrusion]{basicmath} % disable protrusion for tt fonts
}{}
\makeatletter
\@ifundefined{KOMAClassName}{% if non-KOMA class
  \IfFileExists{parskip.sty}{%
    \usepackage{parskip}
  }{% else
    \setlength{\parindent}{0pt}
    \setlength{\parskip}{6pt plus 2pt minus 1pt}}
}{% if KOMA class
  \KOMAoptions{parskip=half}}
\makeatother
\usepackage{xcolor}
\IfFileExists{xurl.sty}{\usepackage{xurl}}{} % add URL line breaks if available
\IfFileExists{bookmark.sty}{\usepackage{bookmark}}{\usepackage{hyperref}}
\hypersetup{
  pdftitle={Varihedsanalyse\_mini1},
  pdfauthor={Gruppe 1.211},
  hidelinks,
  pdfcreator={LaTeX via pandoc}}
\urlstyle{same} % disable monospaced font for URLs
\usepackage[margin=1in]{geometry}
\usepackage{color}
\usepackage{fancyvrb}
\newcommand{\VerbBar}{|}
\newcommand{\VERB}{\Verb[commandchars=\\\{\}]}
\DefineVerbatimEnvironment{Highlighting}{Verbatim}{commandchars=\\\{\}}
% Add ',fontsize=\small' for more characters per line
\usepackage{framed}
\definecolor{shadecolor}{RGB}{248,248,248}
\newenvironment{Shaded}{\begin{snugshade}}{\end{snugshade}}
\newcommand{\AlertTok}[1]{\textcolor[rgb]{0.94,0.16,0.16}{#1}}
\newcommand{\AnnotationTok}[1]{\textcolor[rgb]{0.56,0.35,0.01}{\textbf{\textit{#1}}}}
\newcommand{\AttributeTok}[1]{\textcolor[rgb]{0.77,0.63,0.00}{#1}}
\newcommand{\BaseNTok}[1]{\textcolor[rgb]{0.00,0.00,0.81}{#1}}
\newcommand{\BuiltInTok}[1]{#1}
\newcommand{\CharTok}[1]{\textcolor[rgb]{0.31,0.60,0.02}{#1}}
\newcommand{\CommentTok}[1]{\textcolor[rgb]{0.56,0.35,0.01}{\textit{#1}}}
\newcommand{\CommentVarTok}[1]{\textcolor[rgb]{0.56,0.35,0.01}{\textbf{\textit{#1}}}}
\newcommand{\ConstantTok}[1]{\textcolor[rgb]{0.00,0.00,0.00}{#1}}
\newcommand{\ControlFlowTok}[1]{\textcolor[rgb]{0.13,0.29,0.53}{\textbf{#1}}}
\newcommand{\DataTypeTok}[1]{\textcolor[rgb]{0.13,0.29,0.53}{#1}}
\newcommand{\DecValTok}[1]{\textcolor[rgb]{0.00,0.00,0.81}{#1}}
\newcommand{\DocumentationTok}[1]{\textcolor[rgb]{0.56,0.35,0.01}{\textbf{\textit{#1}}}}
\newcommand{\ErrorTok}[1]{\textcolor[rgb]{0.64,0.00,0.00}{\textbf{#1}}}
\newcommand{\ExtensionTok}[1]{#1}
\newcommand{\FloatTok}[1]{\textcolor[rgb]{0.00,0.00,0.81}{#1}}
\newcommand{\FunctionTok}[1]{\textcolor[rgb]{0.00,0.00,0.00}{#1}}
\newcommand{\ImportTok}[1]{#1}
\newcommand{\InformationTok}[1]{\textcolor[rgb]{0.56,0.35,0.01}{\textbf{\textit{#1}}}}
\newcommand{\KeywordTok}[1]{\textcolor[rgb]{0.13,0.29,0.53}{\textbf{#1}}}
\newcommand{\NormalTok}[1]{#1}
\newcommand{\OperatorTok}[1]{\textcolor[rgb]{0.81,0.36,0.00}{\textbf{#1}}}
\newcommand{\OtherTok}[1]{\textcolor[rgb]{0.56,0.35,0.01}{#1}}
\newcommand{\PreprocessorTok}[1]{\textcolor[rgb]{0.56,0.35,0.01}{\textit{#1}}}
\newcommand{\RegionMarkerTok}[1]{#1}
\newcommand{\SpecialCharTok}[1]{\textcolor[rgb]{0.00,0.00,0.00}{#1}}
\newcommand{\SpecialStringTok}[1]{\textcolor[rgb]{0.31,0.60,0.02}{#1}}
\newcommand{\StringTok}[1]{\textcolor[rgb]{0.31,0.60,0.02}{#1}}
\newcommand{\VariableTok}[1]{\textcolor[rgb]{0.00,0.00,0.00}{#1}}
\newcommand{\VerbatimStringTok}[1]{\textcolor[rgb]{0.31,0.60,0.02}{#1}}
\newcommand{\WarningTok}[1]{\textcolor[rgb]{0.56,0.35,0.01}{\textbf{\textit{#1}}}}
\usepackage{graphicx,grffile}
\makeatletter
\def\maxwidth{\ifdim\Gin@nat@width>\linewidth\linewidth\else\Gin@nat@width\fi}
\def\maxheight{\ifdim\Gin@nat@height>\textheight\textheight\else\Gin@nat@height\fi}
\makeatother
% Scale images if necessary, so that they will not overflow the page
% margins by default, and it is still possible to overwrite the defaults
% using explicit options in \includegraphics[width, height, ...]{}
\setkeys{Gin}{width=\maxwidth,height=\maxheight,keepaspectratio}
% Set default figure placement to htbp
\makeatletter
\def\fps@figure{htbp}
\makeatother
\setlength{\emergencystretch}{3em} % prevent overfull lines
\providecommand{\tightlist}{%
  \setlength{\itemsep}{0pt}\setlength{\parskip}{0pt}}
\setcounter{secnumdepth}{-\maxdimen} % remove section numbering

\title{Varihedsanalyse\_mini1}
\author{Gruppe 1.211}
\date{}

\begin{document}
\maketitle

\hypertarget{data}{%
\section{Data}\label{data}}

\begin{Shaded}
\begin{Highlighting}[]
\KeywordTok{library}\NormalTok{(survival)}\CommentTok{# the package for durvival functions.}
\end{Highlighting}
\end{Shaded}

\begin{verbatim}
## Warning: package 'survival' was built under R version 4.0.3
\end{verbatim}

\begin{Shaded}
\begin{Highlighting}[]
\KeywordTok{library}\NormalTok{(ggfortify) }\CommentTok{# Usefull for intepreting Survival functions plots.}
\end{Highlighting}
\end{Shaded}

\begin{verbatim}
## Warning: package 'ggfortify' was built under R version 4.0.3
\end{verbatim}

\begin{verbatim}
## Loading required package: ggplot2
\end{verbatim}

\begin{Shaded}
\begin{Highlighting}[]
\NormalTok{Cdata<-}\KeywordTok{read.table}\NormalTok{(}\StringTok{"C_data.txt"}\NormalTok{,}\DataTypeTok{header =} \OtherTok{FALSE}\NormalTok{) }\CommentTok{#loading our data}
\KeywordTok{head}\NormalTok{(Cdata) }
\end{Highlighting}
\end{Shaded}

\begin{verbatim}
##    V1 V2 V3 V4 V5 V6 V7 V8
## 1 204  1  4  0  1  2 66 50
## 2 374  1  5  0  0  2 66 43
## 3 184  1 13  1  1  2 57 43
## 4  60  1 16  1  0  2 67 40
## 5 449  1 21  1  1  2 62 25
## 6 224  0 22  0  0  0 75 94
\end{verbatim}

\begin{Shaded}
\begin{Highlighting}[]
\KeywordTok{colnames}\NormalTok{(Cdata) <-}\KeywordTok{c}\NormalTok{(}\StringTok{"Patient_number"}\NormalTok{, }\StringTok{"Death_Censoring"}\NormalTok{, }\StringTok{"Time"}\NormalTok{, }\StringTok{"Treatment"}\NormalTok{, }\StringTok{"Sex"}\NormalTok{, }\StringTok{"Ascites"}\NormalTok{, }\StringTok{"Age"}\NormalTok{, }\StringTok{"Prothrombin"}\NormalTok{) }\CommentTok{#adding names}
\KeywordTok{head}\NormalTok{(Cdata)}
\end{Highlighting}
\end{Shaded}

\begin{verbatim}
##   Patient_number Death_Censoring Time Treatment Sex Ascites Age Prothrombin
## 1            204               1    4         0   1       2  66          50
## 2            374               1    5         0   0       2  66          43
## 3            184               1   13         1   1       2  57          43
## 4             60               1   16         1   0       2  67          40
## 5            449               1   21         1   1       2  62          25
## 6            224               0   22         0   0       0  75          94
\end{verbatim}

\hypertarget{opgave-1}{%
\section{Opgave 1}\label{opgave-1}}

\begin{Shaded}
\begin{Highlighting}[]
\NormalTok{sur.fit1 <-}\StringTok{ }\KeywordTok{survfit}\NormalTok{(}\KeywordTok{Surv}\NormalTok{(Time, Death_Censoring) }\OperatorTok{~}\StringTok{ }\NormalTok{Treatment, }\DataTypeTok{data=}\NormalTok{Cdata)}\CommentTok{# constructing Survival function}
\KeywordTok{names}\NormalTok{(sur.fit1) }\CommentTok{# checking survival function names}
\end{Highlighting}
\end{Shaded}

\begin{verbatim}
##  [1] "n"         "time"      "n.risk"    "n.event"   "n.censor"  "surv"     
##  [7] "std.err"   "cumhaz"    "std.chaz"  "strata"    "type"      "logse"    
## [13] "conf.int"  "conf.type" "lower"     "upper"     "call"
\end{verbatim}

\begin{Shaded}
\begin{Highlighting}[]
\KeywordTok{autoplot}\NormalTok{(sur.fit1,}\DataTypeTok{xlab=}\StringTok{"Days"}\NormalTok{,}\DataTypeTok{ylab=}\StringTok{"Overall survival probability"}\NormalTok{) }\CommentTok{# a clear plot}
\end{Highlighting}
\end{Shaded}

\includegraphics{Varihedsanalyse_mini1_files/figure-latex/unnamed-chunk-2-1.pdf}

Here the red line is prednison treatment and the blue is placebo
treatment.

Log-rank test

\begin{Shaded}
\begin{Highlighting}[]
\NormalTok{Log_rank_test<-}\KeywordTok{survdiff}\NormalTok{(}\KeywordTok{Surv}\NormalTok{(Time, Death_Censoring) }\OperatorTok{~}\NormalTok{Treatment , }\DataTypeTok{data =}\NormalTok{ Cdata)}
\NormalTok{Log_rank_test }\CommentTok{# results}
\end{Highlighting}
\end{Shaded}

\begin{verbatim}
## Call:
## survdiff(formula = Surv(Time, Death_Censoring) ~ Treatment, data = Cdata)
## 
##               N Observed Expected (O-E)^2/E (O-E)^2/V
## Treatment=0 251      142      149     0.355     0.728
## Treatment=1 237      150      143     0.371     0.728
## 
##  Chisq= 0.7  on 1 degrees of freedom, p= 0.4
\end{verbatim}

\begin{Shaded}
\begin{Highlighting}[]
\DecValTok{1} \OperatorTok{-}\StringTok{ }\KeywordTok{pchisq}\NormalTok{(Log_rank_test}\OperatorTok{$}\NormalTok{chisq, }\KeywordTok{length}\NormalTok{(Log_rank_test}\OperatorTok{$}\NormalTok{n) }\OperatorTok{-}\StringTok{ }\DecValTok{1}\NormalTok{)}\CommentTok{#True p-value}
\end{Highlighting}
\end{Shaded}

\begin{verbatim}
## [1] 0.3936019
\end{verbatim}

The log-rank test show no significans in Treatment method.

\hypertarget{opgave-2}{%
\section{Opgave 2}\label{opgave-2}}

\begin{Shaded}
\begin{Highlighting}[]
\NormalTok{Sur1 <-}\StringTok{ }\KeywordTok{Surv}\NormalTok{(Cdata}\OperatorTok{$}\NormalTok{Time, Cdata}\OperatorTok{$}\NormalTok{Death_Censoring) }\CommentTok{#defining our survival function}
\end{Highlighting}
\end{Shaded}

Looking at Treatment

\begin{Shaded}
\begin{Highlighting}[]
\NormalTok{Cox.fit1 <-}\StringTok{ }\KeywordTok{coxph}\NormalTok{(Sur1}\OperatorTok{~}\StringTok{ }\KeywordTok{strata}\NormalTok{(Treatment)}\OperatorTok{+}\StringTok{ }\KeywordTok{factor}\NormalTok{(Sex)}\OperatorTok{+}\StringTok{ }\KeywordTok{factor}\NormalTok{(Ascites)}\OperatorTok{+}\StringTok{ }\NormalTok{Age}\OperatorTok{+}\StringTok{ }\NormalTok{Prothrombin, }\DataTypeTok{data =}\NormalTok{ Cdata) }\CommentTok{# Fitting a Cox model}
\KeywordTok{summary}\NormalTok{(Cox.fit1)}
\end{Highlighting}
\end{Shaded}

\begin{verbatim}
## Call:
## coxph(formula = Sur1 ~ strata(Treatment) + factor(Sex) + factor(Ascites) + 
##     Age + Prothrombin, data = Cdata)
## 
##   n= 488, number of events= 292 
## 
##                       coef exp(coef)  se(coef)      z Pr(>|z|)    
## factor(Sex)1      0.517727  1.678209  0.127220  4.070 4.71e-05 ***
## factor(Ascites)1  0.439503  1.551935  0.180240  2.438   0.0148 *  
## factor(Ascites)2  0.966760  2.629412  0.182675  5.292 1.21e-07 ***
## Age               0.048929  1.050145  0.006810  7.185 6.74e-13 ***
## Prothrombin      -0.012629  0.987450  0.002916 -4.331 1.49e-05 ***
## ---
## Signif. codes:  0 '***' 0.001 '**' 0.01 '*' 0.05 '.' 0.1 ' ' 1
## 
##                  exp(coef) exp(-coef) lower .95 upper .95
## factor(Sex)1        1.6782     0.5959    1.3078    2.1535
## factor(Ascites)1    1.5519     0.6444    1.0901    2.2095
## factor(Ascites)2    2.6294     0.3803    1.8381    3.7614
## Age                 1.0501     0.9522    1.0362    1.0643
## Prothrombin         0.9875     1.0127    0.9818    0.9931
## 
## Concordance= 0.692  (se = 0.018 )
## Likelihood ratio test= 126.1  on 5 df,   p=<2e-16
## Wald test            = 128.9  on 5 df,   p=<2e-16
## Score (logrank) test = 138.4  on 5 df,   p=<2e-16
\end{verbatim}

\begin{Shaded}
\begin{Highlighting}[]
\KeywordTok{source}\NormalTok{(}\StringTok{"andersen.R"}\NormalTok{)}
\NormalTok{test1<-}\KeywordTok{survfit}\NormalTok{(Cox.fit1)}
\NormalTok{test1}
\end{Highlighting}
\end{Shaded}

\begin{verbatim}
## Call: survfit(formula = Cox.fit1)
## 
##               n events median 0.95LCL 0.95UCL
## Treatment=0 251    142   1866    1374    2376
## Treatment=1 237    150   1588    1185    1978
\end{verbatim}

\begin{Shaded}
\begin{Highlighting}[]
\KeywordTok{andersen.plot}\NormalTok{(test1)}
\end{Highlighting}
\end{Shaded}

\includegraphics{Varihedsanalyse_mini1_files/figure-latex/unnamed-chunk-5-1.pdf}

Andersen plots must have strata, plots the cumulative hazard function
for the levels of the strata variable The cumulative hazard looks good.

\begin{Shaded}
\begin{Highlighting}[]
\NormalTok{res1<-}\KeywordTok{residuals}\NormalTok{(Cox.fit1,}\DataTypeTok{type=}\StringTok{"deviance"}\NormalTok{) }\CommentTok{#finding the residual}
\KeywordTok{hist}\NormalTok{(res1)}
\end{Highlighting}
\end{Shaded}

\includegraphics{Varihedsanalyse_mini1_files/figure-latex/unnamed-chunk-6-1.pdf}

not great but looks good

\begin{Shaded}
\begin{Highlighting}[]
\KeywordTok{boxplot}\NormalTok{(res1}\OperatorTok{~}\NormalTok{Cdata}\OperatorTok{$}\NormalTok{Treatment)}
\end{Highlighting}
\end{Shaded}

\includegraphics{Varihedsanalyse_mini1_files/figure-latex/unnamed-chunk-7-1.pdf}

no Outliers

Looking at sex

\begin{Shaded}
\begin{Highlighting}[]
\NormalTok{Cox.fit2 <-}\StringTok{ }\KeywordTok{coxph}\NormalTok{(Sur1}\OperatorTok{~}\StringTok{ }\KeywordTok{factor}\NormalTok{(Treatment)}\OperatorTok{+}\StringTok{ }\KeywordTok{strata}\NormalTok{(Sex)}\OperatorTok{+}\StringTok{ }\KeywordTok{factor}\NormalTok{(Ascites)}\OperatorTok{+}\StringTok{ }\NormalTok{Age}\OperatorTok{+}\StringTok{ }\NormalTok{Prothrombin, }\DataTypeTok{data =}\NormalTok{ Cdata) }\CommentTok{# Fitting a Cox model}
\KeywordTok{summary}\NormalTok{(Cox.fit2)}
\end{Highlighting}
\end{Shaded}

\begin{verbatim}
## Call:
## coxph(formula = Sur1 ~ factor(Treatment) + strata(Sex) + factor(Ascites) + 
##     Age + Prothrombin, data = Cdata)
## 
##   n= 488, number of events= 292 
## 
##                         coef exp(coef)  se(coef)      z Pr(>|z|)    
## factor(Treatment)1  0.054861  1.056393  0.118279  0.464   0.6428    
## factor(Ascites)1    0.455062  1.576270  0.179616  2.534   0.0113 *  
## factor(Ascites)2    0.977054  2.656619  0.182538  5.353 8.67e-08 ***
## Age                 0.048919  1.050135  0.006785  7.209 5.62e-13 ***
## Prothrombin        -0.012850  0.987233  0.002937 -4.376 1.21e-05 ***
## ---
## Signif. codes:  0 '***' 0.001 '**' 0.01 '*' 0.05 '.' 0.1 ' ' 1
## 
##                    exp(coef) exp(-coef) lower .95 upper .95
## factor(Treatment)1    1.0564     0.9466    0.8378    1.3320
## factor(Ascites)1      1.5763     0.6344    1.1085    2.2414
## factor(Ascites)2      2.6566     0.3764    1.8576    3.7993
## Age                   1.0501     0.9523    1.0363    1.0642
## Prothrombin           0.9872     1.0129    0.9816    0.9929
## 
## Concordance= 0.703  (se = 0.018 )
## Likelihood ratio test= 125  on 5 df,   p=<2e-16
## Wald test            = 129.3  on 5 df,   p=<2e-16
## Score (logrank) test = 140.6  on 5 df,   p=<2e-16
\end{verbatim}

\begin{Shaded}
\begin{Highlighting}[]
\KeywordTok{source}\NormalTok{(}\StringTok{"andersen.R"}\NormalTok{)}
\NormalTok{test2<-}\KeywordTok{survfit}\NormalTok{(Cox.fit2)}
\NormalTok{test2}
\end{Highlighting}
\end{Shaded}

\begin{verbatim}
## Call: survfit(formula = Cox.fit2)
## 
##         n events median 0.95LCL 0.95UCL
## Sex=0 198    111   2455    1950    2994
## Sex=1 290    181   1373    1141    1729
\end{verbatim}

\begin{Shaded}
\begin{Highlighting}[]
\KeywordTok{andersen.plot}\NormalTok{(test2)}
\end{Highlighting}
\end{Shaded}

\includegraphics{Varihedsanalyse_mini1_files/figure-latex/unnamed-chunk-8-1.pdf}

Looks good

\begin{Shaded}
\begin{Highlighting}[]
\NormalTok{res2<-}\KeywordTok{residuals}\NormalTok{(Cox.fit2,}\DataTypeTok{type=}\StringTok{"deviance"}\NormalTok{)}
\KeywordTok{hist}\NormalTok{(res2)}
\end{Highlighting}
\end{Shaded}

\includegraphics{Varihedsanalyse_mini1_files/figure-latex/unnamed-chunk-9-1.pdf}

next to no difference

\begin{Shaded}
\begin{Highlighting}[]
\KeywordTok{boxplot}\NormalTok{(res2}\OperatorTok{~}\NormalTok{Cdata}\OperatorTok{$}\NormalTok{Sex)}
\end{Highlighting}
\end{Shaded}

\includegraphics{Varihedsanalyse_mini1_files/figure-latex/unnamed-chunk-10-1.pdf}

Looks good

Looking at Ascites

\begin{Shaded}
\begin{Highlighting}[]
\NormalTok{Cox.fit3 <-}\StringTok{ }\KeywordTok{coxph}\NormalTok{(Sur1}\OperatorTok{~}\StringTok{ }\KeywordTok{factor}\NormalTok{(Treatment)}\OperatorTok{+}\StringTok{ }\KeywordTok{factor}\NormalTok{(Sex)}\OperatorTok{+}\StringTok{ }\KeywordTok{strata}\NormalTok{(Ascites)}\OperatorTok{+}\StringTok{ }\NormalTok{Age}\OperatorTok{+}\StringTok{ }\NormalTok{Prothrombin, }\DataTypeTok{data =}\NormalTok{ Cdata) }\CommentTok{# Fitting a Cox model}
\KeywordTok{summary}\NormalTok{(Cox.fit3)}
\end{Highlighting}
\end{Shaded}

\begin{verbatim}
## Call:
## coxph(formula = Sur1 ~ factor(Treatment) + factor(Sex) + strata(Ascites) + 
##     Age + Prothrombin, data = Cdata)
## 
##   n= 488, number of events= 292 
## 
##                         coef exp(coef)  se(coef)      z Pr(>|z|)    
## factor(Treatment)1  0.092631  1.097057  0.118711  0.780  0.43521    
## factor(Sex)1        0.481967  1.619256  0.126547  3.809  0.00014 ***
## Age                 0.048120  1.049297  0.006795  7.082 1.42e-12 ***
## Prothrombin        -0.012599  0.987480  0.002929 -4.301 1.70e-05 ***
## ---
## Signif. codes:  0 '***' 0.001 '**' 0.01 '*' 0.05 '.' 0.1 ' ' 1
## 
##                    exp(coef) exp(-coef) lower .95 upper .95
## factor(Treatment)1    1.0971     0.9115    0.8693    1.3844
## factor(Sex)1          1.6193     0.6176    1.2636    2.0751
## Age                   1.0493     0.9530    1.0354    1.0634
## Prothrombin           0.9875     1.0127    0.9818    0.9932
## 
## Concordance= 0.646  (se = 0.02 )
## Likelihood ratio test= 75.12  on 4 df,   p=2e-15
## Wald test            = 70.1  on 4 df,   p=2e-14
## Score (logrank) test = 70.43  on 4 df,   p=2e-14
\end{verbatim}

\begin{Shaded}
\begin{Highlighting}[]
\KeywordTok{source}\NormalTok{(}\StringTok{"andersen.R"}\NormalTok{)}
\NormalTok{test3<-}\KeywordTok{survfit}\NormalTok{(Cox.fit3)}
\NormalTok{test3 }\CommentTok{#note we now have 3 variables for hazard function}
\end{Highlighting}
\end{Shaded}

\begin{verbatim}
## Call: survfit(formula = Cox.fit3)
## 
##             n events median 0.95LCL 0.95UCL
## Ascites=0 386    211   1979    1618    2277
## Ascites=1  54     39   1194     802    2187
## Ascites=2  48     42    686     224    1529
\end{verbatim}

\begin{Shaded}
\begin{Highlighting}[]
\KeywordTok{andersen.plot}\NormalTok{(test3)}
\end{Highlighting}
\end{Shaded}

\includegraphics{Varihedsanalyse_mini1_files/figure-latex/unnamed-chunk-11-1.pdf}

Looks good for all 3

\begin{Shaded}
\begin{Highlighting}[]
\NormalTok{res3<-}\KeywordTok{residuals}\NormalTok{(Cox.fit3,}\DataTypeTok{type=}\StringTok{"deviance"}\NormalTok{)}
\KeywordTok{hist}\NormalTok{(res3)}
\end{Highlighting}
\end{Shaded}

\includegraphics{Varihedsanalyse_mini1_files/figure-latex/unnamed-chunk-12-1.pdf}

We have one noticable high residual variable.

\begin{Shaded}
\begin{Highlighting}[]
\KeywordTok{boxplot}\NormalTok{(res3}\OperatorTok{~}\NormalTok{Cdata}\OperatorTok{$}\NormalTok{Ascites)}
\end{Highlighting}
\end{Shaded}

\includegraphics{Varihedsanalyse_mini1_files/figure-latex/unnamed-chunk-13-1.pdf}

no outliers

Looking at Age

\begin{Shaded}
\begin{Highlighting}[]
\KeywordTok{min}\NormalTok{(Cdata[,}\DecValTok{7}\NormalTok{])}\CommentTok{# youngest person is 17}
\end{Highlighting}
\end{Shaded}

\begin{verbatim}
## [1] 17
\end{verbatim}

\begin{Shaded}
\begin{Highlighting}[]
\KeywordTok{max}\NormalTok{(Cdata[,}\DecValTok{7}\NormalTok{])}\CommentTok{# oldest person is 80}
\end{Highlighting}
\end{Shaded}

\begin{verbatim}
## [1] 80
\end{verbatim}

\begin{Shaded}
\begin{Highlighting}[]
\NormalTok{Age.groups=}\KeywordTok{cut}\NormalTok{(Cdata}\OperatorTok{$}\NormalTok{Age,}\DataTypeTok{breaks=}\KeywordTok{c}\NormalTok{(}\OperatorTok{-}\OtherTok{Inf}\NormalTok{,}\KeywordTok{quantile}\NormalTok{(Cdata}\OperatorTok{$}\NormalTok{Age ,}\DataTypeTok{prob=}\KeywordTok{c}\NormalTok{(}\FloatTok{0.33}\NormalTok{,}\FloatTok{0.66}\NormalTok{)),}\OtherTok{Inf}\NormalTok{)) }\CommentTok{#devides the Age into 3 groups}
\NormalTok{Cox.fit4 <-}\StringTok{ }\KeywordTok{coxph}\NormalTok{(Sur1}\OperatorTok{~}\StringTok{ }\KeywordTok{factor}\NormalTok{(Treatment)}\OperatorTok{+}\StringTok{ }\KeywordTok{factor}\NormalTok{(Sex)}\OperatorTok{+}\StringTok{ }\KeywordTok{factor}\NormalTok{(Ascites)}\OperatorTok{+}\StringTok{ }\KeywordTok{strata}\NormalTok{(Age.groups)}\OperatorTok{+}\StringTok{ }\NormalTok{Prothrombin, }\DataTypeTok{data =}\NormalTok{ Cdata) }\CommentTok{# Fitting a Cox model}
\KeywordTok{summary}\NormalTok{(Cox.fit4)}
\end{Highlighting}
\end{Shaded}

\begin{verbatim}
## Call:
## coxph(formula = Sur1 ~ factor(Treatment) + factor(Sex) + factor(Ascites) + 
##     strata(Age.groups) + Prothrombin, data = Cdata)
## 
##   n= 488, number of events= 292 
## 
##                         coef exp(coef)  se(coef)      z Pr(>|z|)    
## factor(Treatment)1  0.026418  1.026770  0.119345  0.221   0.8248    
## factor(Sex)1        0.497157  1.644041  0.127598  3.896 9.77e-05 ***
## factor(Ascites)1    0.459189  1.582789  0.183471  2.503   0.0123 *  
## factor(Ascites)2    1.046229  2.846896  0.184316  5.676 1.38e-08 ***
## Prothrombin        -0.012052  0.988020  0.002876 -4.191 2.78e-05 ***
## ---
## Signif. codes:  0 '***' 0.001 '**' 0.01 '*' 0.05 '.' 0.1 ' ' 1
## 
##                    exp(coef) exp(-coef) lower .95 upper .95
## factor(Treatment)1     1.027     0.9739    0.8126    1.2974
## factor(Sex)1           1.644     0.6083    1.2803    2.1112
## factor(Ascites)1       1.583     0.6318    1.1047    2.2677
## factor(Ascites)2       2.847     0.3513    1.9837    4.0857
## Prothrombin            0.988     1.0121    0.9825    0.9936
## 
## Concordance= 0.641  (se = 0.019 )
## Likelihood ratio test= 73.49  on 5 df,   p=2e-14
## Wald test            = 78.88  on 5 df,   p=1e-15
## Score (logrank) test = 84.54  on 5 df,   p=<2e-16
\end{verbatim}

\begin{Shaded}
\begin{Highlighting}[]
\KeywordTok{source}\NormalTok{(}\StringTok{"andersen.R"}\NormalTok{)}
\NormalTok{test4<-}\KeywordTok{survfit}\NormalTok{(Cox.fit4)}
\NormalTok{test4 }\CommentTok{# we see our group boundries}
\end{Highlighting}
\end{Shaded}

\begin{verbatim}
## Call: survfit(formula = Cox.fit4)
## 
##               n events median 0.95LCL 0.95UCL
## (-Inf,55.7] 161     68   3078    2467    3611
## (55.7,64]   169    106   1559    1263    2057
## (64, Inf]   158    118    852     572    1169
\end{verbatim}

\begin{Shaded}
\begin{Highlighting}[]
\KeywordTok{andersen.plot}\NormalTok{(test4)}
\end{Highlighting}
\end{Shaded}

\includegraphics{Varihedsanalyse_mini1_files/figure-latex/unnamed-chunk-14-1.pdf}

Looks good

\begin{Shaded}
\begin{Highlighting}[]
\NormalTok{res4<-}\KeywordTok{residuals}\NormalTok{(Cox.fit4,}\DataTypeTok{type=}\StringTok{"deviance"}\NormalTok{)}
\KeywordTok{hist}\NormalTok{(res4)}
\end{Highlighting}
\end{Shaded}

\includegraphics{Varihedsanalyse_mini1_files/figure-latex/unnamed-chunk-15-1.pdf}

\begin{Shaded}
\begin{Highlighting}[]
\KeywordTok{boxplot}\NormalTok{(res4}\OperatorTok{~}\NormalTok{Age.groups)}
\end{Highlighting}
\end{Shaded}

\includegraphics{Varihedsanalyse_mini1_files/figure-latex/unnamed-chunk-15-2.pdf}
no outliers

Looking at Protrombin

\begin{Shaded}
\begin{Highlighting}[]
\KeywordTok{min}\NormalTok{(Cdata[,}\DecValTok{8}\NormalTok{])}\CommentTok{# Lowest 12}
\end{Highlighting}
\end{Shaded}

\begin{verbatim}
## [1] 12
\end{verbatim}

\begin{Shaded}
\begin{Highlighting}[]
\KeywordTok{max}\NormalTok{(Cdata[,}\DecValTok{8}\NormalTok{])}\CommentTok{# Highest 135}
\end{Highlighting}
\end{Shaded}

\begin{verbatim}
## [1] 135
\end{verbatim}

\begin{Shaded}
\begin{Highlighting}[]
\NormalTok{Prothrombin.groups=}\KeywordTok{cut}\NormalTok{(Cdata}\OperatorTok{$}\NormalTok{Prothrombin,}\DataTypeTok{breaks=}\KeywordTok{c}\NormalTok{(}\OperatorTok{-}\OtherTok{Inf}\NormalTok{,}\KeywordTok{quantile}\NormalTok{(Cdata}\OperatorTok{$}\NormalTok{Prothrombin ,}\DataTypeTok{prob=}\KeywordTok{c}\NormalTok{(}\FloatTok{0.33}\NormalTok{,}\FloatTok{0.66}\NormalTok{)),}\OtherTok{Inf}\NormalTok{)) }\CommentTok{#devides the Age into 3 groups}
\NormalTok{Cox.fit5 <-}\StringTok{ }\KeywordTok{coxph}\NormalTok{(Sur1}\OperatorTok{~}\StringTok{ }\KeywordTok{factor}\NormalTok{(Treatment)}\OperatorTok{+}\StringTok{ }\KeywordTok{factor}\NormalTok{(Sex)}\OperatorTok{+}\StringTok{ }\KeywordTok{factor}\NormalTok{(Ascites)}\OperatorTok{+}\StringTok{ }\NormalTok{Age}\OperatorTok{+}\StringTok{ }\KeywordTok{strata}\NormalTok{(Prothrombin.groups), }\DataTypeTok{data =}\NormalTok{ Cdata) }\CommentTok{# Fitting a Cox model}
\KeywordTok{summary}\NormalTok{(Cox.fit5)}
\end{Highlighting}
\end{Shaded}

\begin{verbatim}
## Call:
## coxph(formula = Sur1 ~ factor(Treatment) + factor(Sex) + factor(Ascites) + 
##     Age + strata(Prothrombin.groups), data = Cdata)
## 
##   n= 488, number of events= 292 
## 
##                        coef exp(coef) se(coef)     z Pr(>|z|)    
## factor(Treatment)1 0.100662  1.105902 0.118140 0.852  0.39418    
## factor(Sex)1       0.475374  1.608615 0.126927 3.745  0.00018 ***
## factor(Ascites)1   0.460651  1.585105 0.178184 2.585  0.00973 ** 
## factor(Ascites)2   0.952334  2.591752 0.184248 5.169 2.36e-07 ***
## Age                0.048430  1.049622 0.006825 7.096 1.28e-12 ***
## ---
## Signif. codes:  0 '***' 0.001 '**' 0.01 '*' 0.05 '.' 0.1 ' ' 1
## 
##                    exp(coef) exp(-coef) lower .95 upper .95
## factor(Treatment)1     1.106     0.9042    0.8773     1.394
## factor(Sex)1           1.609     0.6217    1.2543     2.063
## factor(Ascites)1       1.585     0.6309    1.1179     2.248
## factor(Ascites)2       2.592     0.3858    1.8062     3.719
## Age                    1.050     0.9527    1.0357     1.064
## 
## Concordance= 0.672  (se = 0.018 )
## Likelihood ratio test= 91.68  on 5 df,   p=<2e-16
## Wald test            = 90.79  on 5 df,   p=<2e-16
## Score (logrank) test = 93.72  on 5 df,   p=<2e-16
\end{verbatim}

\begin{Shaded}
\begin{Highlighting}[]
\KeywordTok{source}\NormalTok{(}\StringTok{"andersen.R"}\NormalTok{)}
\NormalTok{test5<-}\KeywordTok{survfit}\NormalTok{(Cox.fit5)}
\NormalTok{test5 }\CommentTok{# note our group boundries}
\end{Highlighting}
\end{Shaded}

\begin{verbatim}
## Call: survfit(formula = Cox.fit5)
## 
##             n events median 0.95LCL 0.95UCL
## (-Inf,58] 171    127   1169     883    1582
## (58,80]   165     96   1698    1203    2195
## (80, Inf] 152     69   2364    1976    3132
\end{verbatim}

\begin{Shaded}
\begin{Highlighting}[]
\KeywordTok{andersen.plot}\NormalTok{(test5)}
\end{Highlighting}
\end{Shaded}

\includegraphics{Varihedsanalyse_mini1_files/figure-latex/unnamed-chunk-16-1.pdf}

Looks good

\begin{Shaded}
\begin{Highlighting}[]
\NormalTok{res5<-}\KeywordTok{residuals}\NormalTok{(Cox.fit5,}\DataTypeTok{type=}\StringTok{"deviance"}\NormalTok{)}
\KeywordTok{hist}\NormalTok{(res5)}
\end{Highlighting}
\end{Shaded}

\includegraphics{Varihedsanalyse_mini1_files/figure-latex/unnamed-chunk-17-1.pdf}

\begin{Shaded}
\begin{Highlighting}[]
\KeywordTok{boxplot}\NormalTok{(res5}\OperatorTok{~}\NormalTok{Prothrombin.groups)}
\end{Highlighting}
\end{Shaded}

\includegraphics{Varihedsanalyse_mini1_files/figure-latex/unnamed-chunk-17-2.pdf}

no outliers

\hypertarget{opgave-3}{%
\section{Opgave 3}\label{opgave-3}}

\begin{Shaded}
\begin{Highlighting}[]
\NormalTok{Sur1 <-}\StringTok{ }\KeywordTok{Surv}\NormalTok{(Cdata}\OperatorTok{$}\NormalTok{Time, Cdata}\OperatorTok{$}\NormalTok{Death_Censoring) }\CommentTok{# our survival}
\NormalTok{Cox.fit <-}\StringTok{ }\KeywordTok{coxph}\NormalTok{(Sur1}\OperatorTok{~}\StringTok{ }\KeywordTok{factor}\NormalTok{(Treatment)}\OperatorTok{+}\StringTok{ }\KeywordTok{factor}\NormalTok{(Sex)}\OperatorTok{+}\StringTok{ }\KeywordTok{factor}\NormalTok{(Ascites)}\OperatorTok{+}\StringTok{ }\NormalTok{Age}\OperatorTok{+}\StringTok{ }\NormalTok{Prothrombin, }\DataTypeTok{data =}\NormalTok{ Cdata)}
\NormalTok{Cox.fit}
\end{Highlighting}
\end{Shaded}

\begin{verbatim}
## Call:
## coxph(formula = Sur1 ~ factor(Treatment) + factor(Sex) + factor(Ascites) + 
##     Age + Prothrombin, data = Cdata)
## 
##                         coef exp(coef)  se(coef)      z        p
## factor(Treatment)1  0.067070  1.069371  0.117868  0.569   0.5693
## factor(Sex)1        0.525289  1.690948  0.126974  4.137 3.52e-05
## factor(Ascites)1    0.441588  1.555174  0.179236  2.464   0.0138
## factor(Ascites)2    0.974063  2.648684  0.181719  5.360 8.31e-08
## Age                 0.049151  1.050378  0.006764  7.266 3.70e-13
## Prothrombin        -0.012923  0.987161  0.002917 -4.431 9.39e-06
## 
## Likelihood ratio test=129.5  on 6 df, p=< 2.2e-16
## n= 488, number of events= 292
\end{verbatim}

\begin{Shaded}
\begin{Highlighting}[]
\KeywordTok{summary}\NormalTok{(Cox.fit)}
\end{Highlighting}
\end{Shaded}

\begin{verbatim}
## Call:
## coxph(formula = Sur1 ~ factor(Treatment) + factor(Sex) + factor(Ascites) + 
##     Age + Prothrombin, data = Cdata)
## 
##   n= 488, number of events= 292 
## 
##                         coef exp(coef)  se(coef)      z Pr(>|z|)    
## factor(Treatment)1  0.067070  1.069371  0.117868  0.569   0.5693    
## factor(Sex)1        0.525289  1.690948  0.126974  4.137 3.52e-05 ***
## factor(Ascites)1    0.441588  1.555174  0.179236  2.464   0.0138 *  
## factor(Ascites)2    0.974063  2.648684  0.181719  5.360 8.31e-08 ***
## Age                 0.049151  1.050378  0.006764  7.266 3.70e-13 ***
## Prothrombin        -0.012923  0.987161  0.002917 -4.431 9.39e-06 ***
## ---
## Signif. codes:  0 '***' 0.001 '**' 0.01 '*' 0.05 '.' 0.1 ' ' 1
## 
##                    exp(coef) exp(-coef) lower .95 upper .95
## factor(Treatment)1    1.0694     0.9351    0.8488    1.3473
## factor(Sex)1          1.6909     0.5914    1.3184    2.1688
## factor(Ascites)1      1.5552     0.6430    1.0945    2.2098
## factor(Ascites)2      2.6487     0.3775    1.8550    3.7819
## Age                   1.0504     0.9520    1.0365    1.0644
## Prothrombin           0.9872     1.0130    0.9815    0.9928
## 
## Concordance= 0.696  (se = 0.017 )
## Likelihood ratio test= 129.5  on 6 df,   p=<2e-16
## Wald test            = 132.3  on 6 df,   p=<2e-16
## Score (logrank) test = 142.4  on 6 df,   p=<2e-16
\end{verbatim}

we look at the p-value of treatment and see that it is not significant,
resulting in Treatment not affecting the result.

\hypertarget{opgave-4}{%
\section{Opgave 4}\label{opgave-4}}

We pull out different variables to see the effect, leveled varables are
not of intresed since all information have be extracted.

Looking at the model without Age

\begin{Shaded}
\begin{Highlighting}[]
\NormalTok{Sur1 <-}\StringTok{ }\KeywordTok{Surv}\NormalTok{(Cdata}\OperatorTok{$}\NormalTok{Time, Cdata}\OperatorTok{$}\NormalTok{Death_Censoring)}
\NormalTok{Cox.fit._age <-}\StringTok{ }\KeywordTok{coxph}\NormalTok{(Sur1}\OperatorTok{~}\StringTok{ }\KeywordTok{factor}\NormalTok{(Treatment)}\OperatorTok{+}\StringTok{ }\KeywordTok{factor}\NormalTok{(Sex)}\OperatorTok{+}\StringTok{ }\KeywordTok{factor}\NormalTok{(Ascites)}\OperatorTok{+}\StringTok{ }\NormalTok{Prothrombin, }\DataTypeTok{data =}\NormalTok{ Cdata) }\CommentTok{# Fitting Cox model}
\NormalTok{res_age=}\KeywordTok{residuals}\NormalTok{(Cox.fit._age) }\CommentTok{#residuals}
\KeywordTok{scatter.smooth}\NormalTok{(Cdata}\OperatorTok{$}\NormalTok{Age, res_age)}
\end{Highlighting}
\end{Shaded}

\includegraphics{Varihedsanalyse_mini1_files/figure-latex/unnamed-chunk-19-1.pdf}

Using martingales residuals we see a linear behaviour for Age,
indicating that using Age as a linear term is correct.

Looking at the model without Prothrombin

\begin{Shaded}
\begin{Highlighting}[]
\NormalTok{Sur1 <-}\StringTok{ }\KeywordTok{Surv}\NormalTok{(Cdata}\OperatorTok{$}\NormalTok{Time, Cdata}\OperatorTok{$}\NormalTok{Death_Censoring)}
\NormalTok{Cox.fit._Prothrombin <-}\StringTok{ }\KeywordTok{coxph}\NormalTok{(Sur1}\OperatorTok{~}\StringTok{ }\KeywordTok{factor}\NormalTok{(Treatment)}\OperatorTok{+}\StringTok{ }\KeywordTok{factor}\NormalTok{(Sex)}\OperatorTok{+}\StringTok{ }\KeywordTok{factor}\NormalTok{(Ascites)}\OperatorTok{+}\StringTok{ }\NormalTok{Age, }\DataTypeTok{data =}\NormalTok{ Cdata) }\CommentTok{# Fitting Cox model}
\NormalTok{res_Prothrombin=}\KeywordTok{residuals}\NormalTok{(Cox.fit._Prothrombin) }\CommentTok{#residuals}
\KeywordTok{scatter.smooth}\NormalTok{(Cdata}\OperatorTok{$}\NormalTok{Prothrombin, res_Prothrombin)}
\end{Highlighting}
\end{Shaded}

\includegraphics{Varihedsanalyse_mini1_files/figure-latex/unnamed-chunk-20-1.pdf}

Using martingale residuals we see a decreasing linear behavoiur with a
small curv in the middel, showing that using Prothrombin as a linear
term is correct.

\hypertarget{opgave-5}{%
\section{Opgave 5}\label{opgave-5}}

\begin{Shaded}
\begin{Highlighting}[]
\NormalTok{Sur1 <-}\StringTok{ }\KeywordTok{Surv}\NormalTok{(Cdata}\OperatorTok{$}\NormalTok{Time, Cdata}\OperatorTok{$}\NormalTok{Death_Censoring) }\CommentTok{#our survival function}
\end{Highlighting}
\end{Shaded}

Interaction Treatment and Sex

\begin{Shaded}
\begin{Highlighting}[]
\NormalTok{Cox.fit_TS <-}\StringTok{ }\KeywordTok{coxph}\NormalTok{(Sur1}\OperatorTok{~}\StringTok{ }\KeywordTok{factor}\NormalTok{(Treatment)}\OperatorTok{*}\KeywordTok{factor}\NormalTok{(Sex)}\OperatorTok{+}\StringTok{ }\KeywordTok{factor}\NormalTok{(Ascites)}\OperatorTok{+}\StringTok{ }\NormalTok{Age}\OperatorTok{+}\StringTok{ }\NormalTok{Prothrombin, }\DataTypeTok{data =}\NormalTok{ Cdata)}
\NormalTok{Cox.fit_TS}
\end{Highlighting}
\end{Shaded}

\begin{verbatim}
## Call:
## coxph(formula = Sur1 ~ factor(Treatment) * factor(Sex) + factor(Ascites) + 
##     Age + Prothrombin, data = Cdata)
## 
##                                      coef exp(coef)  se(coef)      z        p
## factor(Treatment)1               0.220814  1.247091  0.192829  1.145 0.252157
## factor(Sex)1                     0.653601  1.922450  0.181024  3.611 0.000306
## factor(Ascites)1                 0.443838  1.558678  0.179358  2.475 0.013339
## factor(Ascites)2                 0.967216  2.630611  0.182275  5.306 1.12e-07
## Age                              0.049307  1.050543  0.006805  7.245 4.32e-13
## Prothrombin                     -0.012752  0.987329  0.002927 -4.357 1.32e-05
## factor(Treatment)1:factor(Sex)1 -0.246543  0.781497  0.243803 -1.011 0.311901
## 
## Likelihood ratio test=130.6  on 7 df, p=< 2.2e-16
## n= 488, number of events= 292
\end{verbatim}

\begin{Shaded}
\begin{Highlighting}[]
\KeywordTok{summary}\NormalTok{(Cox.fit_TS)}
\end{Highlighting}
\end{Shaded}

\begin{verbatim}
## Call:
## coxph(formula = Sur1 ~ factor(Treatment) * factor(Sex) + factor(Ascites) + 
##     Age + Prothrombin, data = Cdata)
## 
##   n= 488, number of events= 292 
## 
##                                      coef exp(coef)  se(coef)      z Pr(>|z|)
## factor(Treatment)1               0.220814  1.247091  0.192829  1.145 0.252157
## factor(Sex)1                     0.653601  1.922450  0.181024  3.611 0.000306
## factor(Ascites)1                 0.443838  1.558678  0.179358  2.475 0.013339
## factor(Ascites)2                 0.967216  2.630611  0.182275  5.306 1.12e-07
## Age                              0.049307  1.050543  0.006805  7.245 4.32e-13
## Prothrombin                     -0.012752  0.987329  0.002927 -4.357 1.32e-05
## factor(Treatment)1:factor(Sex)1 -0.246543  0.781497  0.243803 -1.011 0.311901
##                                    
## factor(Treatment)1                 
## factor(Sex)1                    ***
## factor(Ascites)1                *  
## factor(Ascites)2                ***
## Age                             ***
## Prothrombin                     ***
## factor(Treatment)1:factor(Sex)1    
## ---
## Signif. codes:  0 '***' 0.001 '**' 0.01 '*' 0.05 '.' 0.1 ' ' 1
## 
##                                 exp(coef) exp(-coef) lower .95 upper .95
## factor(Treatment)1                 1.2471     0.8019    0.8546     1.820
## factor(Sex)1                       1.9225     0.5202    1.3482     2.741
## factor(Ascites)1                   1.5587     0.6416    1.0967     2.215
## factor(Ascites)2                   2.6306     0.3801    1.8404     3.760
## Age                                1.0505     0.9519    1.0366     1.065
## Prothrombin                        0.9873     1.0128    0.9817     0.993
## factor(Treatment)1:factor(Sex)1    0.7815     1.2796    0.4846     1.260
## 
## Concordance= 0.695  (se = 0.017 )
## Likelihood ratio test= 130.6  on 7 df,   p=<2e-16
## Wald test            = 132.1  on 7 df,   p=<2e-16
## Score (logrank) test = 142.7  on 7 df,   p=<2e-16
\end{verbatim}

We see a p-value of 0.311 for Treamtent and Sex menaning no significant
interaction is present.

Interaction Treatment and Ascites

\begin{Shaded}
\begin{Highlighting}[]
\NormalTok{Cox.fit_TA <-}\StringTok{ }\KeywordTok{coxph}\NormalTok{(Sur1}\OperatorTok{~}\StringTok{ }\KeywordTok{factor}\NormalTok{(Sex)}\OperatorTok{+}\KeywordTok{factor}\NormalTok{(Treatment)}\OperatorTok{*}\KeywordTok{factor}\NormalTok{(Ascites)}\OperatorTok{+}\StringTok{ }\NormalTok{Age}\OperatorTok{+}\StringTok{ }\NormalTok{Prothrombin, }\DataTypeTok{data =}\NormalTok{ Cdata)}
\NormalTok{Cox.fit_TA}
\end{Highlighting}
\end{Shaded}

\begin{verbatim}
## Call:
## coxph(formula = Sur1 ~ factor(Sex) + factor(Treatment) * factor(Ascites) + 
##     Age + Prothrombin, data = Cdata)
## 
##                                          coef exp(coef)  se(coef)      z
## factor(Sex)1                         0.503528  1.654548  0.126989  3.965
## factor(Treatment)1                   0.291419  1.338326  0.139323  2.092
## factor(Ascites)1                     0.786840  2.196445  0.234027  3.362
## factor(Ascites)2                     1.455465  4.286476  0.243010  5.989
## Age                                  0.046595  1.047697  0.006718  6.936
## Prothrombin                         -0.013059  0.987025  0.002968 -4.400
## factor(Treatment)1:factor(Ascites)1 -0.729933  0.481941  0.360030 -2.027
## factor(Treatment)1:factor(Ascites)2 -0.902858  0.405409  0.345164 -2.616
##                                            p
## factor(Sex)1                        7.34e-05
## factor(Treatment)1                  0.036467
## factor(Ascites)1                    0.000773
## factor(Ascites)2                    2.11e-09
## Age                                 4.03e-12
## Prothrombin                         1.08e-05
## factor(Treatment)1:factor(Ascites)1 0.042619
## factor(Treatment)1:factor(Ascites)2 0.008904
## 
## Likelihood ratio test=139.1  on 8 df, p=< 2.2e-16
## n= 488, number of events= 292
\end{verbatim}

\begin{Shaded}
\begin{Highlighting}[]
\KeywordTok{summary}\NormalTok{(Cox.fit_TA)}
\end{Highlighting}
\end{Shaded}

\begin{verbatim}
## Call:
## coxph(formula = Sur1 ~ factor(Sex) + factor(Treatment) * factor(Ascites) + 
##     Age + Prothrombin, data = Cdata)
## 
##   n= 488, number of events= 292 
## 
##                                          coef exp(coef)  se(coef)      z
## factor(Sex)1                         0.503528  1.654548  0.126989  3.965
## factor(Treatment)1                   0.291419  1.338326  0.139323  2.092
## factor(Ascites)1                     0.786840  2.196445  0.234027  3.362
## factor(Ascites)2                     1.455465  4.286476  0.243010  5.989
## Age                                  0.046595  1.047697  0.006718  6.936
## Prothrombin                         -0.013059  0.987025  0.002968 -4.400
## factor(Treatment)1:factor(Ascites)1 -0.729933  0.481941  0.360030 -2.027
## factor(Treatment)1:factor(Ascites)2 -0.902858  0.405409  0.345164 -2.616
##                                     Pr(>|z|)    
## factor(Sex)1                        7.34e-05 ***
## factor(Treatment)1                  0.036467 *  
## factor(Ascites)1                    0.000773 ***
## factor(Ascites)2                    2.11e-09 ***
## Age                                 4.03e-12 ***
## Prothrombin                         1.08e-05 ***
## factor(Treatment)1:factor(Ascites)1 0.042619 *  
## factor(Treatment)1:factor(Ascites)2 0.008904 ** 
## ---
## Signif. codes:  0 '***' 0.001 '**' 0.01 '*' 0.05 '.' 0.1 ' ' 1
## 
##                                     exp(coef) exp(-coef) lower .95 upper .95
## factor(Sex)1                           1.6545     0.6044    1.2900    2.1221
## factor(Treatment)1                     1.3383     0.7472    1.0185    1.7585
## factor(Ascites)1                       2.1964     0.4553    1.3884    3.4748
## factor(Ascites)2                       4.2865     0.2333    2.6623    6.9016
## Age                                    1.0477     0.9545    1.0340    1.0616
## Prothrombin                            0.9870     1.0131    0.9813    0.9928
## factor(Treatment)1:factor(Ascites)1    0.4819     2.0749    0.2380    0.9760
## factor(Treatment)1:factor(Ascites)2    0.4054     2.4666    0.2061    0.7974
## 
## Concordance= 0.701  (se = 0.017 )
## Likelihood ratio test= 139.1  on 8 df,   p=<2e-16
## Wald test            = 148.8  on 8 df,   p=<2e-16
## Score (logrank) test = 167.7  on 8 df,   p=<2e-16
\end{verbatim}

Because we have 3 levels in Ascites we get 2 varaibles in our model,
each has a significans accounting to the p-value, however it is adviced
to look at the model as a whole and compaire it to the orginal model.

Standard model

\begin{Shaded}
\begin{Highlighting}[]
\NormalTok{Cox.fit <-}\StringTok{ }\KeywordTok{coxph}\NormalTok{(Sur1}\OperatorTok{~}\StringTok{ }\KeywordTok{factor}\NormalTok{(Treatment)}\OperatorTok{+}\StringTok{ }\KeywordTok{factor}\NormalTok{(Sex)}\OperatorTok{+}\StringTok{ }\KeywordTok{factor}\NormalTok{(Ascites)}\OperatorTok{+}\StringTok{ }\NormalTok{Age}\OperatorTok{+}\StringTok{ }\NormalTok{Prothrombin, }\DataTypeTok{data =}\NormalTok{ Cdata)}
\NormalTok{Cox.fit}
\end{Highlighting}
\end{Shaded}

\begin{verbatim}
## Call:
## coxph(formula = Sur1 ~ factor(Treatment) + factor(Sex) + factor(Ascites) + 
##     Age + Prothrombin, data = Cdata)
## 
##                         coef exp(coef)  se(coef)      z        p
## factor(Treatment)1  0.067070  1.069371  0.117868  0.569   0.5693
## factor(Sex)1        0.525289  1.690948  0.126974  4.137 3.52e-05
## factor(Ascites)1    0.441588  1.555174  0.179236  2.464   0.0138
## factor(Ascites)2    0.974063  2.648684  0.181719  5.360 8.31e-08
## Age                 0.049151  1.050378  0.006764  7.266 3.70e-13
## Prothrombin        -0.012923  0.987161  0.002917 -4.431 9.39e-06
## 
## Likelihood ratio test=129.5  on 6 df, p=< 2.2e-16
## n= 488, number of events= 292
\end{verbatim}

\begin{Shaded}
\begin{Highlighting}[]
\KeywordTok{summary}\NormalTok{(Cox.fit)}
\end{Highlighting}
\end{Shaded}

\begin{verbatim}
## Call:
## coxph(formula = Sur1 ~ factor(Treatment) + factor(Sex) + factor(Ascites) + 
##     Age + Prothrombin, data = Cdata)
## 
##   n= 488, number of events= 292 
## 
##                         coef exp(coef)  se(coef)      z Pr(>|z|)    
## factor(Treatment)1  0.067070  1.069371  0.117868  0.569   0.5693    
## factor(Sex)1        0.525289  1.690948  0.126974  4.137 3.52e-05 ***
## factor(Ascites)1    0.441588  1.555174  0.179236  2.464   0.0138 *  
## factor(Ascites)2    0.974063  2.648684  0.181719  5.360 8.31e-08 ***
## Age                 0.049151  1.050378  0.006764  7.266 3.70e-13 ***
## Prothrombin        -0.012923  0.987161  0.002917 -4.431 9.39e-06 ***
## ---
## Signif. codes:  0 '***' 0.001 '**' 0.01 '*' 0.05 '.' 0.1 ' ' 1
## 
##                    exp(coef) exp(-coef) lower .95 upper .95
## factor(Treatment)1    1.0694     0.9351    0.8488    1.3473
## factor(Sex)1          1.6909     0.5914    1.3184    2.1688
## factor(Ascites)1      1.5552     0.6430    1.0945    2.2098
## factor(Ascites)2      2.6487     0.3775    1.8550    3.7819
## Age                   1.0504     0.9520    1.0365    1.0644
## Prothrombin           0.9872     1.0130    0.9815    0.9928
## 
## Concordance= 0.696  (se = 0.017 )
## Likelihood ratio test= 129.5  on 6 df,   p=<2e-16
## Wald test            = 132.3  on 6 df,   p=<2e-16
## Score (logrank) test = 142.4  on 6 df,   p=<2e-16
\end{verbatim}

note here that Ascites contains 3 levels, so we compaire Likelihood
ratio tests against the standard model

we use a anova to get the chi squared value for the 2 models, the
difference in likelihood ratio test is 139.1-129.5=9.4 and the DF is
8-6=2

\begin{Shaded}
\begin{Highlighting}[]
\KeywordTok{anova}\NormalTok{(Cox.fit_TA,Cox.fit)}
\end{Highlighting}
\end{Shaded}

\begin{verbatim}
## Analysis of Deviance Table
##  Cox model: response is  Sur1
##  Model 1: ~ factor(Sex) + factor(Treatment) * factor(Ascites) + Age + Prothrombin
##  Model 2: ~ factor(Treatment) + factor(Sex) + factor(Ascites) + Age + Prothrombin
##    loglik Chisq Df P(>|Chi|)   
## 1 -1519.4                      
## 2 -1524.2 9.558  2  0.008404 **
## ---
## Signif. codes:  0 '***' 0.001 '**' 0.01 '*' 0.05 '.' 0.1 ' ' 1
\end{verbatim}

we get at p-value of 0.008 so the interaction between Treatment and
Ascites is significant.

Interaction Treatment and Age

\begin{Shaded}
\begin{Highlighting}[]
\NormalTok{Cox.fit_TAA <-}\StringTok{ }\KeywordTok{coxph}\NormalTok{(Sur1}\OperatorTok{~}\StringTok{ }\KeywordTok{factor}\NormalTok{(Sex)}\OperatorTok{+}\KeywordTok{factor}\NormalTok{(Ascites)}\OperatorTok{+}\StringTok{ }\KeywordTok{factor}\NormalTok{(Treatment)}\OperatorTok{*}\NormalTok{Age}\OperatorTok{+}\StringTok{ }\NormalTok{Prothrombin, }\DataTypeTok{data =}\NormalTok{ Cdata)}
\NormalTok{Cox.fit_TAA}
\end{Highlighting}
\end{Shaded}

\begin{verbatim}
## Call:
## coxph(formula = Sur1 ~ factor(Sex) + factor(Ascites) + factor(Treatment) * 
##     Age + Prothrombin, data = Cdata)
## 
##                             coef exp(coef)  se(coef)      z        p
## factor(Sex)1            0.529397  1.697909  0.126685  4.179 2.93e-05
## factor(Ascites)1        0.462584  1.588172  0.180018  2.570   0.0102
## factor(Ascites)2        0.992521  2.698027  0.182583  5.436 5.45e-08
## factor(Treatment)1     -0.964723  0.381089  0.805313 -1.198   0.2309
## Age                     0.040708  1.041548  0.009283  4.385 1.16e-05
## Prothrombin            -0.013056  0.987029  0.002919 -4.473 7.71e-06
## factor(Treatment)1:Age  0.016710  1.016851  0.012905  1.295   0.1954
## 
## Likelihood ratio test=131.2  on 7 df, p=< 2.2e-16
## n= 488, number of events= 292
\end{verbatim}

\begin{Shaded}
\begin{Highlighting}[]
\KeywordTok{summary}\NormalTok{(Cox.fit_TAA)}
\end{Highlighting}
\end{Shaded}

\begin{verbatim}
## Call:
## coxph(formula = Sur1 ~ factor(Sex) + factor(Ascites) + factor(Treatment) * 
##     Age + Prothrombin, data = Cdata)
## 
##   n= 488, number of events= 292 
## 
##                             coef exp(coef)  se(coef)      z Pr(>|z|)    
## factor(Sex)1            0.529397  1.697909  0.126685  4.179 2.93e-05 ***
## factor(Ascites)1        0.462584  1.588172  0.180018  2.570   0.0102 *  
## factor(Ascites)2        0.992521  2.698027  0.182583  5.436 5.45e-08 ***
## factor(Treatment)1     -0.964723  0.381089  0.805313 -1.198   0.2309    
## Age                     0.040708  1.041548  0.009283  4.385 1.16e-05 ***
## Prothrombin            -0.013056  0.987029  0.002919 -4.473 7.71e-06 ***
## factor(Treatment)1:Age  0.016710  1.016851  0.012905  1.295   0.1954    
## ---
## Signif. codes:  0 '***' 0.001 '**' 0.01 '*' 0.05 '.' 0.1 ' ' 1
## 
##                        exp(coef) exp(-coef) lower .95 upper .95
## factor(Sex)1              1.6979     0.5890   1.32458    2.1765
## factor(Ascites)1          1.5882     0.6297   1.11600    2.2601
## factor(Ascites)2          2.6980     0.3706   1.88639    3.8589
## factor(Treatment)1        0.3811     2.6241   0.07862    1.8472
## Age                       1.0415     0.9601   1.02277    1.0607
## Prothrombin               0.9870     1.0131   0.98140    0.9927
## factor(Treatment)1:Age    1.0169     0.9834   0.99145    1.0429
## 
## Concordance= 0.695  (se = 0.017 )
## Likelihood ratio test= 131.2  on 7 df,   p=<2e-16
## Wald test            = 134.3  on 7 df,   p=<2e-16
## Score (logrank) test = 144.7  on 7 df,   p=<2e-16
\end{verbatim}

We see a p-value of 0.19 for the interaction between Treatment and Age,
meaning it is not significant.

Interaction Treatment and Prothrombin

\begin{Shaded}
\begin{Highlighting}[]
\NormalTok{Cox.fit_TP <-}\StringTok{ }\KeywordTok{coxph}\NormalTok{(Sur1}\OperatorTok{~}\StringTok{ }\KeywordTok{factor}\NormalTok{(Sex)}\OperatorTok{+}\KeywordTok{factor}\NormalTok{(Ascites)}\OperatorTok{+}\StringTok{ }\NormalTok{Age}\OperatorTok{+}\KeywordTok{factor}\NormalTok{(Treatment)}\OperatorTok{*}\NormalTok{Prothrombin, }\DataTypeTok{data =}\NormalTok{ Cdata)}
\NormalTok{Cox.fit_TP}
\end{Highlighting}
\end{Shaded}

\begin{verbatim}
## Call:
## coxph(formula = Sur1 ~ factor(Sex) + factor(Ascites) + Age + 
##     factor(Treatment) * Prothrombin, data = Cdata)
## 
##                                     coef exp(coef)  se(coef)      z        p
## factor(Sex)1                    0.518523  1.679545  0.127462  4.068 4.74e-05
## factor(Ascites)1                0.441740  1.555411  0.179070  2.467 0.013631
## factor(Ascites)2                0.978996  2.661782  0.181549  5.392 6.95e-08
## Age                             0.048780  1.049989  0.006773  7.202 5.93e-13
## factor(Treatment)1             -0.157295  0.854452  0.377810 -0.416 0.677166
## Prothrombin                    -0.014688  0.985419  0.004080 -3.600 0.000318
## factor(Treatment)1:Prothrombin  0.003450  1.003456  0.005522  0.625 0.532106
## 
## Likelihood ratio test=129.9  on 7 df, p=< 2.2e-16
## n= 488, number of events= 292
\end{verbatim}

\begin{Shaded}
\begin{Highlighting}[]
\KeywordTok{summary}\NormalTok{(Cox.fit_TP)}
\end{Highlighting}
\end{Shaded}

\begin{verbatim}
## Call:
## coxph(formula = Sur1 ~ factor(Sex) + factor(Ascites) + Age + 
##     factor(Treatment) * Prothrombin, data = Cdata)
## 
##   n= 488, number of events= 292 
## 
##                                     coef exp(coef)  se(coef)      z Pr(>|z|)
## factor(Sex)1                    0.518523  1.679545  0.127462  4.068 4.74e-05
## factor(Ascites)1                0.441740  1.555411  0.179070  2.467 0.013631
## factor(Ascites)2                0.978996  2.661782  0.181549  5.392 6.95e-08
## Age                             0.048780  1.049989  0.006773  7.202 5.93e-13
## factor(Treatment)1             -0.157295  0.854452  0.377810 -0.416 0.677166
## Prothrombin                    -0.014688  0.985419  0.004080 -3.600 0.000318
## factor(Treatment)1:Prothrombin  0.003450  1.003456  0.005522  0.625 0.532106
##                                   
## factor(Sex)1                   ***
## factor(Ascites)1               *  
## factor(Ascites)2               ***
## Age                            ***
## factor(Treatment)1                
## Prothrombin                    ***
## factor(Treatment)1:Prothrombin    
## ---
## Signif. codes:  0 '***' 0.001 '**' 0.01 '*' 0.05 '.' 0.1 ' ' 1
## 
##                                exp(coef) exp(-coef) lower .95 upper .95
## factor(Sex)1                      1.6795     0.5954    1.3083    2.1562
## factor(Ascites)1                  1.5554     0.6429    1.0950    2.2094
## factor(Ascites)2                  2.6618     0.3757    1.8648    3.7993
## Age                               1.0500     0.9524    1.0361    1.0640
## factor(Treatment)1                0.8545     1.1703    0.4075    1.7918
## Prothrombin                       0.9854     1.0148    0.9776    0.9933
## factor(Treatment)1:Prothrombin    1.0035     0.9966    0.9927    1.0144
## 
## Concordance= 0.695  (se = 0.017 )
## Likelihood ratio test= 129.9  on 7 df,   p=<2e-16
## Wald test            = 132.4  on 7 df,   p=<2e-16
## Score (logrank) test = 142.5  on 7 df,   p=<2e-16
\end{verbatim}

A p-value of 0.532 results in a no significants between Treatment and
Prothrombin.

\hypertarget{opgave-6}{%
\section{Opgave 6}\label{opgave-6}}

\begin{Shaded}
\begin{Highlighting}[]
\NormalTok{Sur1 <-}\StringTok{ }\KeywordTok{Surv}\NormalTok{(Cdata}\OperatorTok{$}\NormalTok{Time, Cdata}\OperatorTok{$}\NormalTok{Death_Censoring) }\CommentTok{# our survival function}
\NormalTok{Cox.fit <-}\StringTok{ }\KeywordTok{coxph}\NormalTok{(Sur1}\OperatorTok{~}\StringTok{ }\KeywordTok{factor}\NormalTok{(Treatment)}\OperatorTok{+}\StringTok{ }\KeywordTok{factor}\NormalTok{(Sex)}\OperatorTok{+}\StringTok{ }\KeywordTok{factor}\NormalTok{(Ascites)}\OperatorTok{+}\StringTok{ }\NormalTok{Age}\OperatorTok{+}\StringTok{ }\NormalTok{Prothrombin, }\DataTypeTok{data =}\NormalTok{ Cdata) }\CommentTok{#Cox model}
\NormalTok{Cox.fit}
\end{Highlighting}
\end{Shaded}

\begin{verbatim}
## Call:
## coxph(formula = Sur1 ~ factor(Treatment) + factor(Sex) + factor(Ascites) + 
##     Age + Prothrombin, data = Cdata)
## 
##                         coef exp(coef)  se(coef)      z        p
## factor(Treatment)1  0.067070  1.069371  0.117868  0.569   0.5693
## factor(Sex)1        0.525289  1.690948  0.126974  4.137 3.52e-05
## factor(Ascites)1    0.441588  1.555174  0.179236  2.464   0.0138
## factor(Ascites)2    0.974063  2.648684  0.181719  5.360 8.31e-08
## Age                 0.049151  1.050378  0.006764  7.266 3.70e-13
## Prothrombin        -0.012923  0.987161  0.002917 -4.431 9.39e-06
## 
## Likelihood ratio test=129.5  on 6 df, p=< 2.2e-16
## n= 488, number of events= 292
\end{verbatim}

\begin{Shaded}
\begin{Highlighting}[]
\KeywordTok{summary}\NormalTok{(Cox.fit)}
\end{Highlighting}
\end{Shaded}

\begin{verbatim}
## Call:
## coxph(formula = Sur1 ~ factor(Treatment) + factor(Sex) + factor(Ascites) + 
##     Age + Prothrombin, data = Cdata)
## 
##   n= 488, number of events= 292 
## 
##                         coef exp(coef)  se(coef)      z Pr(>|z|)    
## factor(Treatment)1  0.067070  1.069371  0.117868  0.569   0.5693    
## factor(Sex)1        0.525289  1.690948  0.126974  4.137 3.52e-05 ***
## factor(Ascites)1    0.441588  1.555174  0.179236  2.464   0.0138 *  
## factor(Ascites)2    0.974063  2.648684  0.181719  5.360 8.31e-08 ***
## Age                 0.049151  1.050378  0.006764  7.266 3.70e-13 ***
## Prothrombin        -0.012923  0.987161  0.002917 -4.431 9.39e-06 ***
## ---
## Signif. codes:  0 '***' 0.001 '**' 0.01 '*' 0.05 '.' 0.1 ' ' 1
## 
##                    exp(coef) exp(-coef) lower .95 upper .95
## factor(Treatment)1    1.0694     0.9351    0.8488    1.3473
## factor(Sex)1          1.6909     0.5914    1.3184    2.1688
## factor(Ascites)1      1.5552     0.6430    1.0945    2.2098
## factor(Ascites)2      2.6487     0.3775    1.8550    3.7819
## Age                   1.0504     0.9520    1.0365    1.0644
## Prothrombin           0.9872     1.0130    0.9815    0.9928
## 
## Concordance= 0.696  (se = 0.017 )
## Likelihood ratio test= 129.5  on 6 df,   p=<2e-16
## Wald test            = 132.3  on 6 df,   p=<2e-16
## Score (logrank) test = 142.4  on 6 df,   p=<2e-16
\end{verbatim}

\begin{Shaded}
\begin{Highlighting}[]
\NormalTok{new.surf<-}\KeywordTok{survfit}\NormalTok{(Cox.fit,}\DataTypeTok{newdata =} \KeywordTok{data.frame}\NormalTok{(}\DataTypeTok{Treatment=}\DecValTok{0}\NormalTok{,}\DataTypeTok{Sex=}\DecValTok{1}\NormalTok{,}\DataTypeTok{Age=}\DecValTok{57}\NormalTok{,}\DataTypeTok{Ascites=}\DecValTok{1}\NormalTok{,}\DataTypeTok{Prothrombin=}\DecValTok{85}\NormalTok{))}
\end{Highlighting}
\end{Shaded}

our survival function for the specified person

\begin{Shaded}
\begin{Highlighting}[]
\NormalTok{new.age=(}\DecValTok{65-57}\NormalTok{)}\OperatorTok{*}\DecValTok{365}
\end{Highlighting}
\end{Shaded}

number of days untill the person id 65 years old

\begin{Shaded}
\begin{Highlighting}[]
\KeywordTok{autoplot}\NormalTok{(new.surf,}\DataTypeTok{xlab=}\StringTok{"Days"}\NormalTok{,}\DataTypeTok{ylab=}\StringTok{"Overall survival probability"}\NormalTok{)}\OperatorTok{+}
\StringTok{  }\KeywordTok{geom_vline}\NormalTok{(}\DataTypeTok{xintercept =}\NormalTok{ new.age)}
\end{Highlighting}
\end{Shaded}

\includegraphics{Varihedsanalyse_mini1_files/figure-latex/unnamed-chunk-30-1.pdf}

Using the plot we see that the probability of survival after 2920 days,
at 65 years old, is approximately 25\%.

\end{document}
